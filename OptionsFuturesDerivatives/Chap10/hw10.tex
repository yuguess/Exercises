\documentclass[12pt]{article}%
\usepackage{amsfonts}
\usepackage{fancyhdr}
\usepackage{comment}
\usepackage[a4paper, top=2.5cm, bottom=2.5cm, left=2.2cm, right=2.2cm]%
{geometry}
\usepackage{times}
\usepackage{amsmath}
\usepackage{changepage}
\usepackage{amssymb}
\usepackage{graphicx}%
\setcounter{MaxMatrixCols}{30}
\newtheorem{theorem}{Theorem}
\newtheorem{acknowledgement}[theorem]{Acknowledgement}
\newtheorem{algorithm}[theorem]{Algorithm}
\newtheorem{axiom}{Axiom}
\newtheorem{case}[theorem]{Case}
\newtheorem{claim}[theorem]{Claim}
\newtheorem{conclusion}[theorem]{Conclusion}
\newtheorem{condition}[theorem]{Condition}
\newtheorem{conjecture}[theorem]{Conjecture}
\newtheorem{corollary}[theorem]{Corollary}
\newtheorem{criterion}[theorem]{Criterion}
\newtheorem{definition}[theorem]{Definition}
\newtheorem{example}[theorem]{Example}
\newtheorem{exercise}[theorem]{Exercise}
\newtheorem{lemma}[theorem]{Lemma}
\newtheorem{notation}[theorem]{Notation}
\newtheorem{problem}[theorem]{Problem}
\newtheorem{proposition}[theorem]{Proposition}
\newtheorem{remark}[theorem]{Remark}
\newtheorem{solution}[theorem]{Solution}
\newtheorem{summary}[theorem]{Summary}
\newenvironment{proof}[1][Proof]{\textbf{#1.} }{\ \rule{0.5em}{0.5em}}

\newcommand{\Q}{\mathbb{Q}}
\newcommand{\R}{\mathbb{R}}
\newcommand{\C}{\mathbb{C}}
\newcommand{\Z}{\mathbb{Z}}

\begin{document}

\title{Chapter 10}
\author{Dalong Cheng}
\date{\today}
\maketitle
\section*{10.1}

\section*{10.2}

\section*{10.3}

\section*{10.4}
Because selling an option will have possibility to lose money in the future,
so margin is required to avoid risk. While buying an option will only have 
possibility to gain profit, so margin is usually not required.

\section*{10.5}
  \subsection{a}
  April 1, will trade Apr, May, Jul, Oct expiration options

  \subsection{b}
  May 30, will trade, Jun, Jul, Aug, Nov

\section*{10.6}
The strike price will adjust to 60/2=30, and the call option 
covered shares will also be doubled

\section*{10.7}
The exercise of option add more equity to company, thus 
influence the capital structure, the exchange traded equity does not has the same effect.

\section*{10.8}
pros: exchange traded options contract has standard clearing process, 
which garantted it will not default. 
cons: OTC option has relatively larger default risk, while the specifiction
of contract can be tailored to fit issuer's interest, which make it more
flexible

\section*{10.9}
When the stock price goes above 105 at th maturity time.

\section*{10.10}
When the stock price stay above 52, the seller will make profit. 
The put option will be exercised when the stock price go below 60.

\section*{10.11}

\section*{10.12}

\section*{10.13}
Because American option has multiple exercise date, it is equal to an 
european option plus some rights to exercise eariler.

\section*{10.14}
 

\section*{10.15}
When the underlying price go up, long a call and selling a put will 
tend to profit, call's profit potential is unlimited, while put's profit is 
limited by its selling price. When price goes down, put's loss is unlimited

\section*{10.16}


\section*{10.17}

\section*{10.18}


\section*{10.19}

\section*{10.20}

\section*{10.21}

\section*{10.22}

\end{document}
